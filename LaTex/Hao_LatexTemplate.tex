\documentclass[11pt]{article}

%\usepackage{fullpage}
%\usepackage[top=1in, bottom=1in, left=1in, right=1in]{geometry}
\usepackage[margin=1in, paperwidth=8.5in, paperheight=11in]{geometry}
\usepackage{amsfonts}
\usepackage{graphicx}

\def\eq1{y=\frac{x}{3x^2+x+1}}
\def\labelaxes{Remember to include a scale and label your axes.}

\begin{document}

\tableofcontents

\title{My Practice \LaTeX \ Document}
\author{Hao Zhou}
\date{\today}
\maketitle

This is my first Latex document.

\begin{center}
\includegraphics[width=2in,angle=30]{pku.png}

Images must be saved as .png, .jpg, .gif, or .pdf files.
\end{center}

\begin{figure}[htbp]
\centering
\includegraphics[width=.3\textwidth]{pku.png}
\caption{Cool!}
\label{fig:pku}
Figure \ref{fig:pku} shows you Hao's feeling.
\end{figure}

Suppose we are given a rectangle with side
lengths $(x+1)$ and $(x+3)$. Then the equation
$A=x^2+4x+3$ represents the area of the rectangle.

Suppose we are given a rectangle with side
lengths $(x+1)$ and $(x+3)$. Then the equation
$$A=x^2+4x+3$$ represents the area of the rectangle.

Superscripts: $$2x^3$$
$$2x^{34}$$
$$2x^{3x+4}$$
$$2x^{3x^4+5}$$

Subscripts:
$$x_1$$
$$x_{12}$$
$${{x_1}_2}_3$$

Greek letters:
$$\pi$$
$$\alpha$$
$$A=\pi r^2$$

Special letters:

The set of Natural numbers is denoted by $\mathbb{N}$.

The set of Integers is denoted by $\mathbb{Z}$.

The set of Real numbers is denoted by $\mathbb{R}$.


Trig functions:
$$y=\sin{x}$$

Log functions:
$$\log_5{x}$$
$$\ln{x}$$

Square roots:
$$\sqrt{2}$$
$$\sqrt[3]{2}$$
$$\sqrt{x^2+y^2}$$
$$\sqrt{1+\sqrt{x}}$$

Fractions:

About ${\frac{2}{3}}$ of the glass is full.

About $\displaystyle{\frac{2}{3}}$ of the glass is full.

$$\frac{x}{x^2+x+1}$$
$$\frac{\sqrt{x+1}}{\sqrt{x-1}}$$
$$\frac{1}{1+\frac{1}{x}}$$
$$\sqrt{\frac{x}{x^2+x+1}}$$

Brackets:
$$(x+1)$$
$$3[2+(x+1)]$$
$$\{a,b,c\}$$
$$\$12.55$$
$$3(\frac{2}{5})$$
$$3\left(\frac{2}{5}\right)$$
$$3\left[\frac{2}{5}\right]$$
$$3\left\{\frac{2}{5}\right\}$$
$$|\frac{x}{x+1}|$$
$$\left| \frac{x}{x+1} \right|$$
$$\left\{x^2\right\}$$
$$\left\{x^2\right.$$

Calculus Notation:

The function $f(x)=(x-3)^2+\frac{1}{2}$ has domain $\mathrm{D}_f:(-\infty,\infty)$ and range $\mathrm{R}_f:\left[\frac{1}{2},\infty\right)$. \\

$\left. \frac{dy}{dx} \right|_{x=1}$ \\

$\lim \limits_{x \to a^-} f(x)$
$\lim_{x \to a}$ \\

$\displaystyle{\lim \limits_{x \to a} \frac{f(x)-f(a)}{x-a}=f'(a)}$ \\

$\int \sin x \,dx=-\cos x+C$
$\displaystyle{\int \sin x \,dx=-\cos x+C}$ \\

$\int_a^b$
$\int \limits_a^b$ \\

$\displaystyle{\int_a^b}$
$\displaystyle{\int \limits_a^b}$\\

$\displaystyle{\int \limits_{a}^{b} x^2 \,dx=\left.\frac{x^3}{3}\right|_a^b=\frac{b^3}{3}-\frac{a^3}{3}}$\\

$\displaystyle{\sum \limits_{n=1}^{\infty} ar^n = a+ar+ar^2+\cdots}$\\

$\displaystyle{\int_a^b f(x) \,dx=\lim \limits_{x \to \infty} \sum \limits_{k=1}^{n}f(x_k)\cdot \Delta x}$ \\

$\vec{v}=v_1 \vec{i}+v_2\vec{j}=\langle v_1,v_2\rangle$\\

Table:\\

\begin{tabular}{|c|c|c|c|c|c|}
\hline
$x$ & 1 & 2 & 3 & 4 & 5 \\ \hline
$f(x)$ & 10 & 11 & 12 & 13 & 14 \\ \hline
\end{tabular} \\

\[
\left.
\begin{array}{c|c||c}
x_{11} & x_{12} & \ldots \\
\hline
x_{21} & x_{22} & \ldots \\
\hline\hline
\vdots & \vdots & \ddots
\end{array}
\right)
\]

\begin{table}[htbp]
\centering
\begin{tabular}{|r||l|}
    \hline
    $x$ & $x^2$ \\
    \hline \hline
    1 & 1  \\ \hline
    2 & 4  \\ \cline{1-1}
    3 & 9  \\ \hline
    4 & 16 \\ \hline
    \multicolumn{2}{|c|}{...}\\ \hline
\end{tabular}
\caption{Value of $x^2$}
\label{tab:square}
\end{table}
Table \ref{tab:square} is of nonsense....

Equation:

\begin{eqnarray}
x^2-9&=&x+3\\
4x^2&=&12\\
x^3&=&3\\
x&\approx&\pm1.372
\end{eqnarray}

\begin{eqnarray*}
x^2-9&=&x+3\\
4x^2&=&12\\
x^3&=&3\\
x&\approx&\pm1.372
\end{eqnarray*}

Hao says that
\begin{equation}
a+b=c
\label{eq:abc}
\end{equation}
is a philosophical theorem.

Equation (\ref{eq:abc}) is Hao's philosophical theorem.

Enumerate:

\begin{enumerate}
\item pencil
\item paper
\item calculator
\item ruler
\item notebook
	\begin{enumerate}
	\item assessments
		\begin{enumerate}
		\item tests
		\item quizzes
		\end{enumerate}
	\item homework
	\item notes
	\end{enumerate}
\item graph paper
\end{enumerate}

\begin{itemize}
\item pencil
\item paper
\item calculator
\item ruler
\item notebook
	\begin{itemize}
	\item assessments
		\begin{itemize}
		\item tests
		\item quizzes
		\end{itemize}
	\item homework
	\item notes
	\end{itemize}
\item graph paper
\end{itemize}

\begin{enumerate}
\item[Commutative] $a+b=b+a$
\item[Associative] $a+(b+c)=(a+b)+c$
\item[Distributive] $a(b+c)=ab+bc$
\end{enumerate}

Front size:

This will produce \textit{italicized} text.

This will produce \textbf{bold-faced} text.

This will produce \textsc{small caps} text.

This will produce \texttt{typewriter} font.

Please visit Hao Zhou's website at \texttt{http://xxxxxx.com}

Please excuse my dear aunt Sally.

Please excuse my \begin{large}dear aunt Sally\end{large}.

Please excuse my \begin{Large}dear aunt Sally\end{Large}.

Please excuse my \begin{LARGE}dear aunt Sally\end{LARGE}. 

Please excuse my \begin{huge}dear aunt Sally\end{huge}.

Please excuse my \begin{Huge}dear aunt Sally\end{Huge}.

Please excuse my \begin{small}dear aunt Sally\end{small}.

Please excuse my \begin{tiny}dear aunt Sally\end{tiny}.

Format:

\begin{center}This is entered.\end{center}
\begin{flushleft}This is left-justified.\end{flushleft}
\begin{flushright}This is right-justified.\end{flushright}

Quote:

Graph $\eq1$.

Identify the asymptotes of for the graph of $\eq1$. 

\labelaxes\\

Sections:

\section{Linear Functions}
	\subsection{Slope-Intercept Form}
	The slope-intercept form of a linear function is given by $y=ax+b$.
	\subsection{Standard From}
	\subsection{Point-Slope Form}
\section{Quadratic Functions}
	\subsection{Vertex Form}
	\subsection{Standard Form}
	\subsection{Factored Form}

\end{document}